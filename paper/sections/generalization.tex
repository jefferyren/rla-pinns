Essential to the formula above is that the discrete bilinear form (not explicitly stated above) can be written in the way
\begin{equation}\label{eq:gramian_asssumption}
  a^h(u,v)
  =
  \langle T(u), T(v) \rangle_{\mathbb R^N}.
\end{equation}
This implies
\begin{equation*}
  F(\theta) = D(T\circ P)^T \cdot D(T \circ P).
\end{equation*}
If you compute the entries of $F$ this yields
\begin{equation*}
  F(\theta)_{ij} = \langle T(\partial_{\theta_i}u_\theta), T(\partial_{\theta_j}u_\theta) \rangle_{\mathbb R^N}.
\end{equation*}
In the case of the Laplacian this $G$ will be one of the summands, so for example $F_\Omega$.
The map $T$ needs to be linear and is a combination of PDE operators and point evaluation, in the case of the Laplacian
\begin{equation*}
  T:C^2(\Omega) \to \mathbb R^{N_\Omega}, \quad T(u) = \frac{1}{\sqrt{N_{\Omega}}}(\Delta u(x_1), \dots, \Delta u(x_{N_{\Omega}})).
\end{equation*}
The map $P$ is the parametrization, i.e., $\theta\mapsto u_\theta$.
I think the form \eqref{eq:gramian_asssumption} is true for all of the PDEs we have in mind...but Johannes should double check that.

%%% Local Variables:
%%% mode: latex
%%% TeX-master: "../main"
%%% End:
